% Please do not change the document class
\documentclass{scrartcl}

% Please do not change these packages
\usepackage[hidelinks]{hyperref}
\usepackage[none]{hyphenat}
\usepackage{setspace}
\doublespace

% You may add additional packages here
\usepackage{amsmath}
\usepackage{graphicx}
\usepackage{wrapfig}
\graphicspath{ {./images/} }

% Please include a clear, concise, and descriptive title
\title{Graphics Research Journal} 

% Please do not change the subtitle
\subtitle{COMP220- Research Journal}

% Please put your student number in the author field
\author{1507516}

\begin{document}

\maketitle

\abstract{}

\section{Introduction}
For my graphics and simulation project I have chosen to make a voxel based game, where the player is a light source that will light up the terrain. The terrain will be generated using perlin noise \cite{perlin1985image} to change the height of the ground.
%An overview of what the paper is about

\textbf{Paper One:}
Perlin Noise Pixel Shaders
\cite{hart2001perlin}
\par

This paper demonstrates how they used procedural shading techniques to compute high resolution textures efficiently in real time. This means that materials like wood and stone can be generated using procedural texturing to quickly produce dynamic animated enviroments.

They use the perlin noise function to generate these procedural shading techniques \cite{perlin2002improving}. The paper also illustrates the pros and cons of generating the noise algorithm on different graphics hardware.

\par




\textbf{Paper Two:}
Improving Noise
\cite{perlin2002improving}
\par

This paper is what I used to implement the perlin noise algorithm into my project. It work by calculating a random vector for each of the nearest verticies in a cube.
This paper is improves upon his previous paper \textit{An Image Synthesizer} \cite{perlin1985image}.

\par 





\textbf{Paper Three:}
The Multilevel Finite Element Method for Adaptive Mesh
Optimization and Visualization of Volume Data
\cite{grosso1997multilevel}
\par

This paper is about optimising a 2 or 3 dimensional mesh. 


Rendering large amounts  of geometric primitives (i.e. cubes or spheres etc..) can be very compute intensive, this paper presents two algorithms that show significantly increased rendering quality and speed of these geometic primitives from large datasets.

It does this by generating approximate meshes and then refining the meshes iteratively. 

The mesh is refined based on a sequence of approximations that are adaptively generated by the algorithm.

This approach can be used on very large datasets of complex meshes to generate a lower poly mesh from, i.e. a low poly approximate height map can be generated iteratively from a large topographical height map.

This method could be very useful in games to adaptively reduce the Level of Detail in models as they got further away from the camera.






\textbf{Paper Four:}
Dual/primal mesh optimization for polygonized implicit surfaces
\cite{ohtake2002dual}
\par

\textbf{Paper Five:}
Illumination for computer generated pictures
\cite{phong1975illumination}
\par

This paper was published in June 1975, so it is fairly dated now, but the principles behind it still work fine for todays standsards. However there are improved lighting models, but are generally more complex.






\par
\textbf{Paper Six:}
An improved illumination model for shaded display
\cite{whitted2005improved}
\par





\textbf{Paper Seven:}
Texturing techniques for terrain visualization
\cite{dollner2000texturing}
\par





\textbf{Paper Eight:}
Hypertexture
\cite{perlin1989hypertexture}
\par

Another paper by Ken Perlin.

\par
\textbf{Paper Nine:}
An Image Synthesizer
\cite{perlin1985image}
\par

The original paper by ken perlin, this paper has been cited over 1900 times.

\bibliographystyle{ieeetr}
\bibliography{comp220_Research_Journal}

\end{document}
