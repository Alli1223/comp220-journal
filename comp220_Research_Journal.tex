% Please do not change the document class
\documentclass{scrartcl}

% Please do not change these packages
\usepackage[hidelinks]{hyperref}
\usepackage[none]{hyphenat}
\usepackage{setspace}
\doublespace

% You may add additional packages here
\usepackage{amsmath}
\usepackage{graphicx}
\usepackage{wrapfig}
\graphicspath{ {./images/} }

% Please include a clear, concise, and descriptive title
\title{Graphics Research Journal} 

% Please do not change the subtitle
\subtitle{COMP220- Research Journal}

% Please put your student number in the author field
\author{1507516}

\begin{document}

\maketitle

\abstract{}

\section{Introduction}
This paper will be covering three different game components that went into the Graphics and Simulation project. 
These components are \textit{Procedural Terrain Generation}, \textit{Optimisation of Complex Meshes} and \textit{Illumination}.


% Topics
%: Why is the paper significant and/or influential?
% Why did the researchers choose the approach that they did?
%Is there anything counterintuitive or surprising in the paper?
% Do you disagree with any of the assumptions or claims it makes?
% Does the paper suggest any further research questions?


\section{Procedural Terrain Generation}

\subsection{An Image Synthesizer \cite{perlin1985image}}
\par

This paper by ken perlin has been influential to the generation of more realistic Computer Generated Imagery (CGI) as it has cited over 1900 times and has been integrated into various software packages such as Autodesk \textit{Maya} and \textit{3D Studio Max} \cite{PerlinWebsite}.
\par

This paper describes the concept of a ``Pixel Stream Editor'' (PSE) which lets designes work with a high level programming enviroment to produce realistic CGI.

The PSE allows operations to be performed asynchronously at different pixels, which means that this computation can be performed on a Graphics Processing Unit to speed up rendering times.

However this paper has been improved upon in his paper \textit{Improving Noise} \cite{perlin2002improving}.


\subsection{Improving Noise \cite{perlin2002improving}}
This paper revises some issues that arose in his previous paper that was introduced 17 years before. 
The original noise algorithm suffered from two defects; 

\begin{itemize}
\item {Discontinuity across coordinate-aligned integer boundaries.}
\item  {An expensive and problematic method of computing the gradient.}
\end{itemize}

Both these issues were removed in the improved algorithm proposed in this paper.
\par 


\subsection{Hypertexture \cite{perlin1989hypertexture}}

This paper focuses on the rendering of an intermediate region between object and non-object called ``Hypertexture''.  Examples of this can be complex 3D textures such as hair, fur or smoke.
This paper is based of Ken Perlins previous paper \cite{perlin1985image}.

 Hypertexture objects have no well defined boundaries, the paper proposes a function that describes how the objects texture transitions between the outside and inside of the object.





\section{Optimization}

\subsection{Perlin Noise Pixel Shaders \cite{hart2001perlin}}

This paper demonstrates how they used procedural shading techniques to compute high resolution textures efficiently in real time. This means that materials like wood and stone can be generated using procedural texturing to quickly produce dynamic animated environments.

They use the perlin noise function to generate these procedural shading techniques \cite{perlin2002improving}. The paper also illustrates the pros and cons of generating the noise algorithm on different graphics hardware.
\par

\subsection{The Multilevel Finite Element Method for Adaptive Mesh Optimization and Visualization of Volume Data \cite{grosso1997multilevel}}


This paper is about optimizing a 2 or 3 dimensional mesh. 

Rendering large amounts  of geometric primitives (i.e. cubes or spheres etc..) can be very compute intensive, this paper presents two algorithms that show significantly increased rendering quality and speed of these geometric primitives from large datasets.

It does this by generating approximate meshes and then refining the meshes iteratively. 

The mesh is refined based on a sequence of approximations that are adaptively generated by the algorithm.

This approach can be used on very large datasets of complex meshes to generate a lower poly mesh from, i.e. a low poly approximate height map can be generated iteratively from a large topographical height map.

This method could be very useful in games to adaptively reduce the Level of Detail in models as they got further away from the camera.


Other Papers on this:
\cite{carey1981mesh}










\section{Illumination}
\textbf{Illumination for computer generated pictures \cite{phong1975illumination}}

This paper was published in June 1975, but the principles behind it still work well for 3D video game graphics, where performance is important. However for more realistic lighting effects for 3D Design, ray tracing is a slower, but more accurate solution to complex 3D lighting \cite{Shirley:2005:RayTracing}.

Phong's paper breaks up the lighting calculations into three separate parts; \textit{specular reflection}, \textit{diffuse} and \textit{ambient}.



\subsection{Models of Light Reflection for Computer Synthesized Pictures \cite{Blinn1977}}
This paper by Blinn builds upon Phongs paper \cite{phong1975illumination}. As it has ``a noticle effect primarly for non-metallic and edge lit objects'' \cite{Blinn1977}. 


\subsection{An improved illumination model for shaded display \cite{whitted2005improved}}

This paper presents a shading model that uses global information to calculate intensities, then provides extensions to the ray tracing visible surface algorithm. 
This paper builds upon the work done by Phong \cite{phong1975illumination} and Blinn \cite{Blinn1977}. It has improved the lighing effects, however it not perfect.





\par

\textbf{Paper Seven:}
Texturing techniques for terrain visualization
\cite{dollner2000texturing}
\par


\par


\bibliographystyle{ieeetr}
\bibliography{comp220_Research_Journal}

\end{document}
