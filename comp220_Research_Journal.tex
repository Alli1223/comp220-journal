% Please do not change the document class
\documentclass{scrartcl}

% Please do not change these packages
\usepackage[hidelinks]{hyperref}
\usepackage[none]{hyphenat}
\usepackage{setspace}
\doublespace

% You may add additional packages here
\usepackage{amsmath}
\usepackage{graphicx}
\usepackage{wrapfig}
\graphicspath{ {./images/} }

% Please include a clear, concise, and descriptive title
\title{Graphics Research Journal} 

% Please do not change the subtitle
\subtitle{COMP220- Research Journal}

% Please put your student number in the author field
\author{1507516}

\begin{document}

\maketitle

\abstract{}

\section{Introduction}
For my graphics and simulation project I have chosen to make a voxel based game, where the player is a light source that will light up the terrain. The terrain will be generated using perlin noise to change the height of the ground.


\textbf{Paper One:}
Perlin Noise Pixel Shaders
\cite{hart2001perlin}
\par

This paper demonstrates how they used procedural shading techniques to compute high resolution textures efficiently in real time. This means that materials like wood and stone can be generated using procedural texturing to quickly produce dynamic animated enviroments.

They use the perlin noise function to generate these procedural shading techniques \cite{perlin2002improving}.

\par

\textbf{Paper Two:}
Improving Noise
\cite{perlin2002improving}
\par

This paper is what I used to implement the perlin noise algorithm into my project. It work by calculating a random vector for each of the nearest verticies in a cube.
This paper is improving upon his last paper.

\par 

\textbf{Paper Three:}
The Multilevel Finite Element Method for Adaptive Mesh
Optimization and Visualization of Volume Data
\cite{grosso1997multilevel}
\par

This paper is about optimising a 2 or 3 dimensional mesh. 

\textbf{Paper Four:}
Dual/primal mesh optimization for polygonized implicit surfaces
\cite{ohtake2002dual}
\par

\textbf{Paper Five:}
Illumination for computer generated pictures
\cite{phong1975illumination}
\par

This paper was published in June 1975, so it is fairly dated now, but the principles behind it still work fine for todays standsards. However there are improved lighting models, but are generally more complex.


\par
\textbf{Paper Six:}
An improved illumination model for shaded display
\cite{whitted2005improved}
\par

\textbf{Paper Seven:}
Texturing techniques for terrain visualization
\cite{dollner2000texturing}
\par

\textbf{Paper Eight:}
Hypertexture
\cite{perlin1989hypertexture}
\par

Another paper by Ken Perlin.

\par


\bibliographystyle{ieeetr}
\bibliography{comp220_Research_Journal}

\end{document}
